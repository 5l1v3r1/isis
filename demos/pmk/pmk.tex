%  $RCSfile: pmk.tex,v $ $Revision: 2.1 $ $Date: 90/07/31 16:17:49 $
\subsection*{Using Parallel Make}
 
Parallel make (pmake) is an ISIS application in which several processors
cooperate to execute a job specified in the language of the standard UNIX make
utility.  This introduction to parallel make assumes some familiarity with
standard make, and with the execution environment of ISIS applications.
 
A pmake job is modeled as a directed acyclic graph (DAG) in which nodes
represent steps in the job, and edges represent data dependencies between
steps.  Each explicit rule in pmake has the form

\begin{verbatim}
     {target}* : {dependency}* <cr>
     {<tab> shell command <cr>}*
\end{verbatim}

The commands specified in a rule are executed sequentially as one step in the
pmake job.  The same is true for each instantiation of an implicit rule.
 
For example, consider the following make file text:

\begin{verbatim}
product: file1 file2
 
file1.o : file1.c file1.h
          cc -c file1.c
 
file1 : file1.o
          rm file1
          cc -o file1 file1.o
 
file2 : file2.c file1.o
          rm file2
          cc -o file2 file2.c file1.o
\end{verbatim}
 
product, file1.o, file1 and file2 appear as targets, file1.c, file2.c file1.h
and file1.o appear as dependencies, and cc and rm shell commands are specified
in the rules.  Since the rm and cc commands under file1 are specified in the
same rule they are recognised as a single step.  Parallel make will always
execute them in sequence. The same is true under file2.
 
Like make, parallel make does not require that all dependency and target names
be files.  Symbolic dependencies ( such as "product" in the example ) may
appear which do not represent specific files.  It is VERY important that ALL
dependencies be carefully specified in the make file(s).  Unlike serial make,
parallel make will feel free to execute whole steps in any order allowed by the
dependencies.
 
In the example, standard make would build file1 followed by file2 because that
is the order in which steps are listed in the "product" rule.  Parallel make
might build file1.o, followed by file2, and end with file1 as that is allowed
by the dependencies specified.

 
\subsection*{Parallel vs. Serial Specifications}

Parallel make executes steps in 2 phases, a serial phase followed by a 
parallel phase.  During the serial phase pmake's activity is exactly like
standard make's activity, with one exception.  When pmake is about to execute
a step in the make file it first checks to see whether the user specified that
it be executed during the parallel phase.  If so, a description of the step
and its dependencies is enqueued in a graph file for later execution. 
Otherwise the step is considered to be a serial step, and is executed
immediately.  Thus pmake's capabilities are a superset of make's capabilities.

The user specifies a parallel rule by preceding the first command in the rule
with a | character.  Any rule without a leading | is executed during the serial
phase of pmake.  Rules with leading | characters are accumulated in a graph
file and executed after all of the serial steps in the entire job are complete.
This means that even though specifications of serial and parallel steps may
be interleaved in the make file, all of the serial steps are executed before
any of the parallel ones.  All of the parallel steps are executed by one
invokation of pmkexec, the parallel execution manager.

Pmake insures correct ordering of steps executed during parallel phase using
dependencies accumulated during the serial phase.  The default implicit
rules do not begin with | characters.  Users should either avoid or modify
implicit rules.
 
In the following example, the rm commands are separated from the cc commands
using symbolic dependencies.  The rm commands are not preceded by | symbols, so
they are executed during serial phase as they are encountered. The cc commands
are accumulated and placed in a graph file for later parallel execution.

\begin{verbatim}
product: file1 file2
 
file1.o : file1.c file1.h
          |cc -c file1.c
 
rmfile1:
          rm file1
 
rmfile2:
          rm file2
 
file1 : file1.o rmfile1
          |cc -o file1 file1.o
 
file2 : file2.c file1.o rmfile2
          |cc -o file2 file2.c file1.o
\end{verbatim}
 
The distinction between serial and parallel steps must be carefully considered
when devising a parallel make specification.  For example, in instances where
one pmake includes another as one of its steps, a completely independent 
invokation occurs.  If the imbedded pmake command used default options the
subordinate would execute its own parallel phase before returning.  

It is also possible to accumulate job steps from multiple make files for
execution during one parallel phase.  Since pmake generates the graph file in
append mode, steps accumulate until the graph file is purged.  Normally pmake
purges its graph file after executing the parallel phase.  The -A option can
be used to supress execution of the parallel phase and the purging of the
graph file.  Subordinate pmakes which use the -A option accumulate job steps
for execution during parallel phase, then just leave them in the graph file.
This allows other pmake instances to continue accumulating steps in the same
graph file for execution during ITS parallel phase. In this case the data
dependencies identified by various subordinates must be sufficient to force
correct execution order.
 
In the following example three make files, one of which invokes the other two,
accomplish the same result as the earlier examples.
 
dir/makefile contains the following:

\begin{verbatim}
product: file1 file2
 
rmfiles:
          rm file1
          rm file2
          rm product.gph
 
file1 : rmfiles
          cd subdir1; pmake -G ../product.gph -A
 
file2 : rmfiles
          cd subdir2; pmake -G ../product.gph -A
 
 
dir/subdir1/makefile contains this:
\end{verbatim}
 
\begin{verbatim}
file1:
          |cc -o file1 file1.o
 
file1.o : file1.c file1.h
          |cc -c file1.c
\end{verbatim}
 
and dir/subdir2/makefile contains this:
 
\begin{verbatim}
file2:
          |cc -o file2 file2.c ../subdir1/file1.o
\end{verbatim}

This pmake is invoked with the command:
          pmake -G product.gph

The use of the -G and -A options in the pmake command cause output from the two
submakes to be appended to the same graph file, product.gph.  The rm commands,
including one to remove the old product.gph, are specified serially, so they
execute during the serial pass instead of being queued on product.gph.  Because
the submakes are accumulated in the same graph file, the file and symbolic
names contained in them are resolved together, allowing them to reference each
other.  Parallel make can interleave the execution of commands from the
submakes constrained only by the specified dependencies.
 
All dependency and target names are evaluated in UNIX style, even if they are
symbolic.  Thus 2 names which yield the same path ( taking current directories
into account where appropriate ) will be treated as the same object by parallel
make even if they arose from different sub-makes.  In the example parallel make
realizes that the reference to ../subdir1/file1.o refers to the same object as
file1.o referenced from the other submake.  Parallel make does not account for
synonyms such as those which result from the ln command, so be sure to refer to
such objects in a consistent manner.
 
Be careful of any conditionals such as "if" or "for" statements used in serial
rules which depend on data created or massaged by parallel rules.  Keep in mind
that we are pulling off a major con job on serial make by deferring execution
of its steps.

 
\subsection*{Parallel Make Flow}
 
The serial phase is executed directly by the command "pmake" on make files
which are compatible with standard make with the exceptions mentioned above.
 
During the serial phase of parallel make, steps to be executed in the parallel
phase are accumulated in a graph file.  The user may specify the name of the
graph file using the -G option in the pmake command.  To cause parallel
execution of the commands stored in the graph pmake automatically executes the
command:
 
    pmkexec product.gph pmk 0 6
where:
\begin{itemize}
\item product.gph is the graph file (pmake.gph is the default)
\item pmk is the name of the isis group to be formed for execution
\item 0 is the isis port number associated with the site view to be used
\item 6 is the maximum number of workstations to be used at any one point
during the job.
\end{itemize}

Pmkexec reads the graph file and passes it to the prescheduler which
evaluates and allocates steps to servers in a hopefully reasonable manner.
After that the ISIS process group which will execute the job is formed.  In the
event that execution does not progress as the prescheduler thought it would
steps are dynamically reallocated ( fought over ) by group members.
 
The identity of the servers assigned to the job is specified in the current
ISIS site view.  Parallel make uses only servers which have no users as
process group members other than the server from which the job was begun.
The user must be sure isis is running on the servers before running pmake.
Pmake uses r\_exec to start server processes on remote sites.
 
The standard output from steps executed by each server is accumulated in a
separate results file.  These are concatenated when the job is completed to
present a complete but artificially ordered history of the job in files in the
directory that was current when pmkexec was run.  Pmake prints job output to
stdout upon completion of the job.  Remember that pmake is not robust in the
isis sense, but pmkexec is.


\subsection*{How to Run a Parallel Make Demo}
\begin{enumerate}

\item Load and make the isis system including the demos directory.

\item cd to ~isis/SUN/demos/pmk, the object directory for parallel make.

\item If they don't aleardy exist, copy qtest*.c and makeqtest from
     ~isis/demos/pmk.

\item If qtest exists, type the command:
     make -f makeqtest clean
     to remove the old version

\item Be sure isis is started on the server you are using.  This demo uses up
     to 5 additional servers if isis is running on them and they have no
     other users.

\item type the command:
     
     ../../bin/pmake -f makeqtest -P<port>

     where <port> is the isis port number.
     This causes the following output when run with only 1 available server:
\end{enumerate}
\begin{verbatim}

isis% ../../bin/pmake -f makeqtest -P 1551
SERIAL PHASE
    |cc -c qtest1.c
    |cc -c qtest2.c
    |cc -c qtest3.c
    |cc -c qtest4.c
    |cc -c qtest5.c
    |cc -o qtest qtest1.o qtest2.o qtest3.o qtest4.o qtest5.o
    |qtest
PARALLEL PHASE

RUNNING /usr/u/isis/DEVEL/SUN/bin/pmkexec pmake.gph pmk 1551 6
RESULTS:
**********************************************************************
STEP 1 : cd /usr/fsys/odin/c/b/isis/DEVEL/SUN/demos/pmk;  cc -c qtest2.c; 
**********************************************************************
STEP 2 : cd /usr/fsys/odin/c/b/isis/DEVEL/SUN/demos/pmk;  cc -c qtest3.c; 
**********************************************************************
STEP 3 : cd /usr/fsys/odin/c/b/isis/DEVEL/SUN/demos/pmk;  cc -c qtest4.c; 
**********************************************************************
STEP 0 : cd /usr/fsys/odin/c/b/isis/DEVEL/SUN/demos/pmk;  cc -c qtest1.c; 
**********************************************************************
STEP 4 : cd /usr/fsys/odin/c/b/isis/DEVEL/SUN/demos/pmk;  cc -c qtest5.c; 
**********************************************************************
STEP 5 : cd /usr/fsys/odin/c/b/isis/DEVEL/SUN/demos/pmk; 
         cc -o qtest qtest1.o qtest2.o qtest3.o qtest4.o qtest5.o; 
**********************************************************************
STEP 6 : cd /usr/fsys/odin/c/b/isis/DEVEL/SUN/demos/pmk;  qtest; 
1 2 3 4 5 5 4 3 2 1
\end{verbatim}

The order of the printed results changes depending on the number of servers.
The command associated with each step number should always be the same as that 
shown here.


\subsection*{How to Run Parallel Make With Your Own Makefiles}
\begin{itemize}

\item Locate the absolute files pmake and pmkexec and make pmake accessible to
     the user and/or directory(s) where you will be running pmake.  Both of
     these files must be in the same directory, i.e. ~isis/SUN/bin.

\item Modify or generate your makefile to indicate which steps should be done
     in parallel.  Remember:
       - pmake requires that dependencies be seperated from commands by a
         <cr> <tab> sequence.  A ";" confuses it.
       - The precious option does nothing
       - Parallel steps begin with a "|"

\item Remove the old graph file if it exists.

\item type the pmake command incorperating the following options in addition to
     those supported by standard make:
\begin{description}
\item[-G] graph file name.  Default: pmake.gph.
\item[-P] ISIS port number. Default: 0.
\item[-S] number of servers. Default: 6.
             A negative port number generates lots of messages indicating
             pmkexec's progress.
\item[-K] keep the report files generated by pmake described below.
             Default: Destroy files at end of job.
\item[-A] accumulate parallel steps in the graph file without executing them.
             Default: execute at end of pmake.
\end{description}

\item Pmake will print a report including the commands executed and their
     standard output when execution is complete.

\item If the -K option was requested, the following files will persist:
\begin{itemize}
\item the graph file containing a description of the parallel steps executed
         in a pmake specific format described below.
\item a copy of the graph file annotated with step execution and server
         queue and speed information.  The name of this file is the name of the
         graph file with a "r" appended to it.
\item a map of way the job was actually executed indicating which steps
         were executed by which servers in what order.  Step numbering is
         generated by pmake and can be converted into command contents using
         the graph file. The name of this file is the name of the graph file
         with "r.sum" appended to it.
\item a file for each server containing the standard output it generated.
         The names of these files are the name of the graph file with ".stdoX"
         appended to it, where X is the server number.  Servers are numbered
         beginning at 0 by pmake.
\end{itemize}
       
\item You can rerun the exact same seqeunce of parallel steps just completed
         using the command:

         pmkexec file group port servers
         
where:
\begin{itemize}
\item file = graph file name
\item group = isis process goup name
\item port = isis port
\item server = number of servers
\end{itemize}

         You can also add servers to a job manually by logging into the desired
         server and typing the command

         pmkexec group port servers

         These are the commands used by pmake and rexec to execute the parallel
         part of a parallel make job.

\item When making pmake, use the C preprocessor variable ABSDIR in the cc
         commands for main.c and pmkexec.c to indicate the directory in which
         pmkexec will reside.  The value of ABSDIR must be a string ( including
         double quotes ) containing the complete path name to the directory.
\end{enumerate}

\subsection*{A Larger Example}

The pmakefiles necessary to make isis are included in the ~isis/demos/pmk
account.  In order to pmake isis these files must be placed in the isis object 
directory hierarchy rooted at some directory, say, isis\_home as follows:

\begin{verbatim}
   pmk directory file         copy to:
      pmake.isis              isis_home/makefile
      pmake.protos            isis_home/protos/makefile
      pmake.clib              isis_home/clib/makefile
      pmake.mlib              isis_home/mlib/makefile
      pmake.util              isis_home/util/makefile
      pmake.demos             isis_home/demos/makefile
\end{verbatim}

Pmake can then be run on the isis code from isis\_home.


\subsection*{Internal Graph File Format}

Pmkexec receives descriptions of parallel steps in a textual "graph file" in
which every line contains a 1 to 6 character keyword beginning in its first
column.  The keyword determines the meaning of the rest of the line.
All communication between pmake and pmkexec, as well as the initial state
specifications for new server processes use this format.  In addition, one or
all servers can be forced to dump thier states into a file of this format 
using the send command in the isis cmd utility.  A single parameter indicating
the name of the file in which the dump should be placed is required.  Each
server will append its pmake server number to the file name specified to
yield the name of the dump file to be generated.  Note that the consistency of
these dumps depends on the type of broadcast used by the send command.

Data items and process steps are defined by lines or sequences of lines in
this file.  Data items and steps are numbered for reference by pmkexec in the
order in which they appear in the graph file.  The following line formats may
appear:

\begin{description}
\item[NEW]

       Indicates the beginning of the graph

\item[SIZE]   steps items servers

       Indicates the number of job steps, data items, and server data
       structures to be allocated for the job description.  This information
       is not essential but its presence allows faster processing of the graph
       file.

\item[IN]     name mod size type

       Indicates an input dependency. - Also used to declare a data item for
       future reference as an input or output dependency.
\begin{description}
\item[name] the name of the file
\item[mod] - indiactes whether or not the file is out of date or has been
             modified  ( "U" for unmodified, "M" for modified ), 
\item[size] - size of the file in bytes
\item[type] - numeric file type indicator - NOT USED
\end{description}
       
\item[OUT]    name mod size type

       Indicates an output file generated during execution of a step.  A given
       data item may only be generated by one step.  If an OUT line appears
       outside of a step definition, subsequent IN lines ( up to the next
       OUT or STEP line) designate input dependencies to be applied to the
       step which generates the OUT file.  This allows dependencies to be
       specified before the command for generating a file is known.
       Parameters same as those of IN.

\item[STEP]   content

       Begins definition of a step for execution.  Also declares that the step
       is in a particular processors queue if it lies between a PROC line
       and an END line.
       content - The bourne shell command which constitutes the step.
    
\item[TIME]   milliseconds

       Indicates the expected duration of the step whose definition is in
       progress.
       milliseconds - the expected duration in milliseconds.

\item[CALCS]  ctime dtime mod status dep

       A set of calculations generated by pmkexec.
\begin{description}
\item[ctime] the length of the critical path from this step to the end of
               the job
\item[dtime] The total compute time of steps which depend on this one.
\item[mod] "U" for no recomputation necessary, "M" for recompute.
\item[status] The completion status of the step:
\begin{enumerate}
\item RAW - CALCS not yet computed
\item EVAL - CALCS computed, but step not allocated to processor
\item ALLOCATED - Step allocated to server but dependencies not
                                 fulfilled
\item SCHEDULED - Step allocated and ready to execute,
                                 or executing
\item SCHED\_LOCK - Step allocated and ready but being transferred
                                 or about to change status
\item DONE - Step execution complete.
\end{enumerate}

\item[dep] The number of input dependencies which have not yet been fulfilled
\end{description}

\item[RSLT] start stop exp\_stop deny

       Statistics resulting from executing or scheduling a step.  All times
       in milliseconds.
\begin{description}
\item[start] the time at which execution was started 
\item[stop] the time at which execution ended
\item[exp\_stop] The time at which execution is expected to end based on
                  simulated execution - NOT USED
\item[deny] The time at which a request for the step was last denied.
\end{description}

       These times vary from one server to another because their clocks are not
       synchronized in any way.  The one time on which all servers should agree
       is (stop - start), as the messages actually exchanged between servers
       contain this quantity as reported by the executing server.

\item[ENV]    env\_string
      
 A UNIX environment variable assignment to be installed before executing
       a step.

       env\_string - Specification of a variable and its valus in the form:
                    variable=value

\item[IN#]    item

       Specifies that a previously encountered data item is an input to a step
       
item - The internally generated number of the data item

\item[OUT#]   item

       A previously encountered data item is output by a step.
      
item - The number of the data item.

\item[STDO]   file

       NOT USED

\item[SPEED]  speed\_ratio

       Indicates the measured speed of the execution of a step, defined to be
       the actual execution time of the step divided by the expected execution
       time.  Also used to indicate the average of the speeds of all steps
       executed by a given processor.
       
speed\_ratio - ((stop - start)/TIME)

\item[DUMDUR] milliseconds

       Indicates that execution of a step is to be simulated using a unix sleep
       command of (milliseconds/1000) seconds instead of actually executing the
       command associated with the step.

\item[END]

       Indicates that the definition of a step is complete

\item[PROC]   name
       
Indicates the beginning of the definition of a server.  Definitian ends
       when another PROC command is encountered.  Any steps defined after a
       PROC are in that server's queue.

       name - The ascii name of the server

\item[RANK]   num
       
Indicates the current rank of a server in an isis process group view

       num - the rank, assigned by isis.

\item[STEP#]  num
       
Indicates that a previously encountered step is in a server's queue

       num - the number of the step

\item[DONE]

       Indicates that the all precedeing steps in a processor's queue have been
       completely executed, and that any subsequent steps have not.

\item[DUMSPD] speed\_ratio
       
Indicates the speed at which any SIMULATED steps on this processor
       should be executed.

       speed\_ratio - factor by which to multiply a step's DUMDUR to obtain
                     the length of the sleep command

\end{description}
\subsection*{PMAKE Source Files}
\begin{description}
\item[Makefile] the standard unix make file for making pmake and pmkexec

\item[check.c] a module of the public domain make utility not modified for pmake

\item[input.c] unmodified public domain make module

\item[macro.c] unmodified public domain make module

\item[pmain.c] the pmake main program - modified public domain module

\item[make.c] the pmake dependency interpreter - modified public domain module

\item[make.h] include for pmake - modified public domain module

\item[makeqtest] pmake file for quick test

\item[pmake.*] pmake files for building isis

\item[pmk.doc] this document

\item[pmkdat.h] data structures for pmkexec

\item[pmkexec.c] parallel execution module - encapsulates all isis interaction

\item[pmkgph.c] DAG data structure manipulation

\item[pmkio.c] graph file input and output

\item[pmklib.c] routines converting make command lines and their dependencies into
           graph file entries.  Performs execution time estimates.

\item[pmklst.c] linked lisk manipulation

\item[pmksched6.c] sixth generation scheduling decision maker

\item[qtest*.c] quick pmake demo/test

\item[reader.c] unmodified public domain make module

\item[rules.c] unmodified public domain make module
